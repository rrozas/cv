%%%%%%%%%%%%%%%%%%%%%%%%%%%%%%%%%%%%%%%%%
% Friggeri Resume/CV
% XeLaTeX Template
% Version 1.0 (5/5/13)
%
% This template has been downloaded from:
% http://www.LaTeXTemplates.com
%
% Original author:
% Adrien Friggeri (adrien@friggeri.net)
% https://github.com/afriggeri/CV
%
% License:
% CC BY-NC-SA 3.0 (http://creativecommons.org/licenses/by-nc-sa/3.0/)
%
% Important notes:
% This template needs to be compiled with XeLaTeX and the bibliography, if used,
% needs to be compiled with biber rather than bibtex.
%
%%%%%%%%%%%%%%%%%%%%%%%%%%%%%%%%%%%%%%%%%
\documentclass[a4paper]{friggeri-cv} % Add 'print' as an option into the square bracket to remove colors from this template for printing
\XeTeXinputencoding "latin1" % codage du document
%\addbibresource{bibliography.bib} % Specify the bibliography file to include publications
\bibliography{bibliography}
\usepackage{fancyhdr}
\fancypagestyle{plain}{
    \fancyhf{} %Clear Everything.
    \fancyfoot[C]{} %Page Number
    \renewcommand{\headrulewidth}{0pt} %0pt for no rule, 2pt thicker etc...
    \renewcommand{\footrulewidth}{0pt}
    \fancyfoot[L]{}
    \fancyfoot[R]{\color{gray}\thepage/\pageref{lastpage}}
    \fancyhead{}
    \fancyhead{}
}

\pagestyle{plain}
%%\fancypagestyle{myfoot}{
%%  \renewcommand{\headrulewidth}{0pt}
%%  \renewcommand{\footrulewidth}{0pt}
%%  \fancyfoot{}}
%\fancypagestyle{myfoot}{%
%\fancyfoot[r]{\color{gray}\thepage/\pageref{lastpage}}}% the parbox is required to ensure alignment with a possible center footer (e.g., as in the casual style)
%\pagestyle{myfoot}%}%
\begin{document}

\header{Rony}{Rozas}{Ing\'{e}nieur R\&D} % Your name and current job title/field

%----------------------------------------------------------------------------------------
%	SIDEBAR SECTION
%----------------------------------------------------------------------------------------

\begin{aside} % In the aside, each new line forces a line break
\section{Contact}
51 Avenue Fran\c{c}ois Mitterrand
94000 Cr\'{e}teil 
~
{\color{red}\mobilephonesymbol} +33 (0) 6 50 27 43 11
{\color{red}\fixedphonesymbol} +33 (0) 1 81 66 86 92
~
{\color{red}\emailsymbol} \href{mailto:rozas.rony@gmail.com}{rozas.rony@gmail.com}
%\href{http://www.ronyrozas.com}{http://www.ronyrozas.com}
%\href{http://facebook.com/rony.rozas}{fb://jsmith}
~
Fran\c{c}ais, 28 ans
\section{Mots Cl\'{e}s}
Optimisation, Planification
Processus stochastiques
Data Mining, Mod\'{e}lisation
R\'{e}seaux Bay\'{e}siens
\end{aside}

%----------------------------------------------------------------------------------------
%	EDUCATION SECTION
%----------------------------------------------------------------------------------------

\section{Formation}

\begin{entrylist}
%------------------------------------------------
\entry
{En cours}
{Doctorat {\normalfont -- Informatique et Math\'e{}matiques appliqu\'e{}es}}
{Universit\'e{} Paris-Est}
{}
%------------------------------------------------
\entry
{2009}
{Master {\normalfont -- IT (Informatique et T\'e{}l\'e{}communications)}}
{Universit\'e{} Paul Sabatier}
{sp\'{e}cialit\'{e} : Intelligence Artificielle}
%------------------------------------------------
\entry
{2007}
{Licence {\normalfont -- STS (Sciences Technologies Sant\'e{})} }
{Universit\'e{} des Antilles et de la Guyane}
{sp\'{e}cialit\'{e} :  Informatique}
%------------------------------------------------
\entry
{2006}
%{Dipl\^{o}me d'\'{e}tudes universitaires g\'{e}n\'{e}rale {\normalfont -- MIAS (Math\'e{}matiques Informatiques et Applications aux Sciences)}}
{Dipl�me d'�tudes universitaires g�n�rale {\normalfont -- MIAS (Math\'e{}matiques Informatiques et Applications aux Sciences)}}
{Universit\'e{} des Antilles et de la Guyane}
{sp\'{e}cialit\'{e} :  Informatique}
%------------------------------------------------
\entry
{2004}
{Classe pr\'{e}paratoire aux grandes \'{e}coles {\normalfont-- MPSI (Math\'e{}matiques Physique et Sciences de l'ing\'e{}nieur)}}
{Lyc\'e{}e G\'e{}n\'e{}ral et Technologique de Baimbridge}

%{2011--2012}
%{Masters {\normalfont of Commerce}}
%{The University of California, Berkeley}
%{\emph{Money Is The Root Of All Evil -- Or Is It?} \\ This thesis explored the idea that money has been the cause of untold anguish and suffering in the world. I found that it has, in fact, not.}
%------------------------------------------------
\end{entrylist}

%----------------------------------------------------------------------------------------
%	WORK EXPERIENCE SECTION
%----------------------------------------------------------------------------------------

\section{Exp\'{e}rience Professionnelle}

\begin{entrylist}
%------------------------------------------------
\entry
{En cours}
{Doctorat}
{Marne-la-Vall\'{e}e, France}
{{\small GRETTIA / IFSTTAR}\\
D\'{e}veloppement d'une technique d'optimisation dynamique de param\`{e}tres de maintenance \`{a} partir de flux de  donn\'{e}es sur la fiabilit\'{e} de syst\`{e}mes multi-composants. L'approche propos�e utilise une mod�lisation probabiliste du syst�me �tudi�. Une application visant � r�duire les risques d'indisponibilit�s en garantissant un niveau de s�curit� �lev� des portes de trains a \'{e}t\'{e} r\'{e}alis\'{e}e, dans le cadre du projet SurFer (Surveillance Ferroviaire) en collaboration avec  \textsc{Bombardier}.}

%------------------------------------------------
\entry
{2013 - 2014}
{Attach\'{e} Temporaire d'Enseignement et de Recherche}
{Cr\'{e}teil, France}
{Universit\'{e} Paris Est Cr\'{e}teil \\
Enseignement de modules de g\'{e}nie logiciel, de programmation orient\'{e}e objet et d'algorithmique.}
%------------------------------------------------
\entry
{2010 - 2013}
{Mission d'enseignement}
{Cr\'{e}teil, France}
{Universit\'{e} Paris Est Cr\'{e}teil \\
Enseignement de modules d'algorithmique, de programmation orient\'{e}e objet en Java et de programmation imp\'{e}rative en C.}
%------------------------------------------------
\entry
{F\'{e}vrier 2009\\\`{a} Juillet 2009}
{Stage Master 2 Recherche}
{Toulouse, France}
{{\small INRA / IRIT} \\
D\'{e}veloppement d'algorithmes pour l'optimisation de pr\'{e}f\'{e}rences multi-crit\`{e}res en utilisant le formalisme des probl\`{e}mes de satisfactions de contraintes pond\'{e}r\'{e}es ( VCSP ).}
%------------------------------------------------
\entry
{Janvier 2009\\\`{a} Avril 2010}
{Concepteur/D\'{e}veloppeur}
{Toulouse, France}
{PizzaCap \\
Conception et r\'{e}alisation d'une application, r\'{e}partie sur plusieurs postes, de prise de commande, en utilisant Windev.}
\end{entrylist}
\begin{entrylist}
%------------------------------------------------
\entry
{F\'{e}vrier 2008\\\`{a} Juin 2008}
{Tutorat}
{Martinique, France}
{Universit\'e{} des Antilles et de la Guyane\\
Tutorat de lyc\'{e}en en math\'{e}matiques et physique}
%------------------------------------------------
\entry
{Juillet 2006}
{Emploi saisonnier}
{Guadeloupe, France}
{A\'{e}roport P\^{o}le Cara\"{i}bes\\
Accueil et information des voyageurs}
%------------------------------------------------
\end{entrylist}

%----------------------------------------------------------------------------------------
%	AWARDS SECTION
%----------------------------------------------------------------------------------------

\section{Comp\'{e}tences}

\subsection{Informatiques}
\begin{entrylist}
%------------------------------------------------
\entryinline
{Langages}
{C++, C, Java, Javascript, PHP, HTML, W-langage (Windev/Webdev)}
%------------------------------------------------
\entryinline
{Graphisme}
{Adobe Creative Suite : Photoshop, Illustrator, InDesign, Flash}
%------------------------------------------------
\entryinline
{SGBD}
{Oracle, SQL, MS Access}
%------------------------------------------------
\entryinline{Syst\`{e}mes}
{Windows, Linux}
%------------------------------------------------
\end{entrylist}

\subsection{Scientifiques}
\begin{entrylist}
%------------------------------------------------
\entryinline
{Outils}
{Matlab, Maple, Octave, R}
%------------------------------------------------
\entryinline
{M�thodes}
{Optimisation : Combinatoire, Multicrit�re, M�thaheuristique, Programmation par contraintes  \\
Processus stochastiques : R\'{e}seaux Bay\'{e}sien, Cha\^{i}ne de Markov \\
Data Mining, Planification, Machine Learning, Mod�lisation math�matique}
%------------------------------------------------
\end{entrylist}

\subsection{Langues}

\begin{entrylist}
%------------------------------------------------
\entryinline{Fran�ais}{Langue maternelle}
\entryinline{Anglais}{Lu, \'{e}crit, parl\'{e}}
\entryinline{Espagnol}{Notions}
\end{entrylist}





%----------------------------------------------------------------------------------------
%	COMMUNICATION SKILLS SECTION
%----------------------------------------------------------------------------------------
%
%\section{communication skills}
%
%\begin{entrylist}
%%------------------------------------------------
%\entry
%{2011}
%{Oral Presentation}
%{California Business Conference}
%{Presented the research I conducted for my Masters of Commerce degree.}
%%------------------------------------------------
%\entry
%{2010}
%{Poster}
%{Annual Business Conference, Oregon}
%{As part of the course work for BUS320, I created a poster analyzing several local businesses and presented this at a conference.}
%%------------------------------------------------
%\end{entrylist}

%----------------------------------------------------------------------------------------
%	INTERESTS SECTION
%----------------------------------------------------------------------------------------

\section{Centres d'int\'{e}r\^{e}t}
\begin{entrylist}
\entryinline{Loisirs}{Volley-ball, roller,  salsa, moto}
\entryinline{Associatif}{Responsable de l'association de danse latine de l'Universit\'{e} de Marne-la-Vall\'{e}e}
\end{entrylist}

%----------------------------------------------------------------------------------------
%	PUBLICATIONS SECTION
%----------------------------------------------------------------------------------------

%\section{Publications}

%\printbibsection{article}{article in peer-reviewed journal} % Print all articles from the bibliography
%\nocite{*}
%\printbibliography[ type=article, title={Articles}, heading=subbibliography]
%\printbibsection{book}{books} % Print all books from the bibliography

%\begin{refsection} % This is a custom heading for those references marked as "inproceedings" but not containing "keyword=france"
%\nocite{*}
%\printbibliography[ type=inproceedings, title={international peer-reviewed conferences/proceedings}, notkeyword={france}, heading=subbibliography]
%\end{refsection}

%\begin{refsection} % This is a custom heading for those references marked as "inproceedings" and containing "keyword=france"

%\printbibliography[heading=subbibliography]
%\end{refsection}

%\printbibsection{misc}{other publications} % Print all miscellaneous entries from the bibliography

%\printbibsection{report}{research reports} % Print all research reports from the bibliography
 	
%----------------------------------------------------------------------------------------

\section{R�f�rences}
\begin{tabular}{p{3 cm}p{7.3cm}p{3.5cm}}
Patrice Aknin & Directeur scientifique � la {\small SNCF} & +33 (0)1 57 23 62 92 \vspace{\parsep}\\%+33 (0)6 07 44 96 13 
Laurent Bouillaut & Charg� de recherche � l'{\small IFSTTAR}& +33 (0)1 81 66 87 16\vspace{\parsep}\\
Nihal Perkergin & Enseignant / chercheur � l'{\small UPEC} &+33 (0)1 45 17 16 44\vspace{\parsep}\\
\end{tabular} 

\label{lastpage}
\end{document}